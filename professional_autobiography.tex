\begin{rSection}{Professional Autobiography}
	\par
	My seismic journey began on a volcano, albeit a formerly submarine, Precambrian one in the Canadian Shield. Before starting my undergraduate degree at Carleton University (Ottawa, Ontario, Canada), I spent a summer trudging through muskeg and deploying geophones near Flin Flon (Manitoba, Canada) for 2D and 3D seismic surveys of the volcanogenic massive sulfide ore deposits that are mined there. This experience inspired me to transfer from General Physics to the Computational Geophysics program in the first year of my undergraduate degree.
	
	\par
	I spent the summer of my junior year helping scientists at the Geological Survey of Canada (GSC) collect seismic and electrical-resistivity data to image hydrogeological systems. For my undergraduate thesis, I processed 2D seismic reflection data from one of these surveys to image a buried valley that, we hypothesized, controlled the flow of groundwater that leeched salt ions from the clay overburden, which subsequently destabilized as a result of ion loss and failed in a landslide.
	
	\par
	During my senior year, I worked at the Pacific Geoscience Center (a subdivision of the GSC) as a research assistant analyzing data from the Canadian National Seismic Network to monitor earthquake and mining activity in Western Canada. Upon graduating, I took a position as a seismic analyst at the Scripps Institution of Oceanography (SIO), where I analyzed data from local, regional, and teleseismic earthquakes recorded by the EarthScope Transportable Array experiment. While at SIO, I also conducted maintenance fieldwork for the Anza Regional Network (in Southern California) and took part in various temporary deployments including one experiment comprising $\sim$1100 nodal seismographs (Ben-Zion \textit{et al}., 2015).
	
	\par
	After three years at SIO, I moved to the University of Southern California to start my Ph.D. with Prof. Yehuda Ben-Zion. I have focused my graduate coursework on mathematics and physics, with an emphasis on computational/numerical analysis, and completed two major research projects, both of which entailed automated processing of large seismic data sets to analyze local seismicity and fault-zone structure in Southern California. As I anticipate graduating in May 2021, I am looking for exciting and challenging opportunities to employ my experience in new environments.
	
\end{rSection}