\begin{rSection}{Graduate Coursework (Selected)}
	\begin{timeline}
		2020 & \textbf{Advanced Mechanics}\\
		& Newtonian formulation of dynamics; Hamilton's principle; Lagrangian formulation; rigid body motion; Hamiltonian formulation; Hamilton-Jacobi theory; vibrations.\\ \\
		& \textbf{Advanced Seismology}\\
		& Advanced methods of theoretical seismology for studying the generation of seismic waves from natural and artificial sources and the propagation through realistic earth models. \\ \\
		& \textbf{Selected Topics in Computational Physics} \\
		& Algorithmic Techniques in Artificial Intelligence and Machine Learning \\ \\
		& \textbf{Numerical Analysis and Computation}\\
		& Linear equations and matrices, Gauss elimination, error estimates, iteration techniques; contractive mappings, Newton's method; matrix eigenvalue problems; least-squares approximation, Newton-Cotes and Gaussian quadratures; finite difference methods.\\ \\
		2017 & \textbf{Probability for Electrical and Computer Engineers} \\ 
		& Rigorous coverage of probability, discrete and continuous random variables, functions of multiple random variables, covariance, correlation, random sequences, Markov chains, estimation, and introduction to statistics. \\ \\
		& \textbf{Methods of Computational Physics} \\ 
		& Introduction to algorithm development. Integration of ordinary differential equations; chaotic systems; molecular dynamics; Monte Carlo integration and simulations; cellular automata and other complex systems. \\ \\
		& \textbf{Introduction to Digital Signal Processing} \\
		& Fundamentals of digital signal processing covering: discrete time linear systems, quantization, sampling, Z-transforms, Fourier transforms, FFTs and filter design. \\ \\
		2016 & \textbf{Methods of Theoretical Physics} \\
		& Vector analysis; infinite, asymptotic Fourier series; complete sets; Dirac delta function; Fourier, Laplace transforms; Legendre functions; spherical harmonics; Sturm-Liouville theory; orthogonal polynomials; gamma-factorial function; complex variables. \\
	\end{timeline}
\end{rSection}